\chapter{Miscellaneous}

\section{Bug reports and suggestions for improvements}
To report bugs or suggest improvements to the code, please send an email to the CIG Computational Seismology Mailing List (cig-seismo@geodynamics.org) or Alessia Maggi (alessia@sismo.u-strasbg.fr), and/or use our online bug tracking system Roundup (www.geodynamics.org/roundup).

\section{Notes and Acknowledgments}
The main developers of the FLEXWIN source code are Alessia Maggi and Carl Tape.  The following individuals (listed in alphabetical order) have also contributed to the development of the source code: Daniel Chao, Min Chen, Vala Hjorleifsdottir, Qinya Liu, Jeroen Tromp.  The following individuals (listed in alphabetical order) contributed to this manual: Sue Kientz, Alessia Maggi, Carl Tape.

The FLEXWIN code makes use of filtering and enveloping algorithms that are part of SAC (Seismic Analysis Code, Lawerence Livermore National Laboratory) provided for free to IRIS members.  We thank Brian Savage for adding interfaces to these algorithms in recent SAC distributions. 

We acknowledge support by the National Science Foundation under grant EAR-0711177.
Daniel Chao received additional support from a California Institute of Technology Summer Undergraduate Reseach Fellowship.


\section{Copyright}
Copyright 2009, by the California Institute of Technology (U.S.) and University of Strasbourg (France).  ALL RIGHTS RESERVED.  U.S. Government Sponsorship Acknowledged.

Any commercial use must be negotiated with the Office of Technology Transfer at
the California Institute of Technology.  This software may be subject to U.S.
export control laws and regulations.  By accepting this software, the user
agrees to comply with all applicable U.S. export las and regulations, including
the International Traffic and Arms Regulations, 22 C.F.R 120-130 and the Export
Administration Regulations, 15 C.F.R. 730-744.  User has the responsibility to
obtain export licenses, or other export authority as may be required before
exporting such information to foreign countries or providing access to foreign
nationals.  In no event shall the California Institute of Technology be liable
to any party for direct, indirect, special, incidental or consequential
damages, including lost profits, arising out of the use of this software and
its documentation, even if the California Institute of Technology has been
advised of the possibility of such damage.

The California Institute of Technology specifically disclaims any warranties,
included the implied warranties or merchantability and fitness for a particular
purpose.  The software and documentation provided hereunder is on an `as is'
basis, and the California Institute of Technology has no obligations to provide
maintenance, support, updates, enhancements or modifications.
